\documentclass[letterpaper,10pt,draftclsnofoot,onecolumn]{IEEEtran}

\usepackage{graphicx}                                        
\usepackage{amssymb}                                         
\usepackage{amsmath}                                         
\usepackage{amsthm}                                          

\usepackage{alltt}                                           
\usepackage{float}
\usepackage{color}
\usepackage{url}

\usepackage{balance}
\usepackage[TABBOTCAP, tight]{subfigure}
\usepackage{enumitem}
\usepackage{pstricks, pst-node}

\usepackage{textcomp}
\usepackage[margin=0.75in]{geometry}

\parindent = 0.0 in
\parskip = 0.1 in

\begin{document}

\title{Problem Statement for\\ Phone Service Alert Application}

\author{
\IEEEauthorblockN{Xiaoyong Zheng}
\IEEEauthorblockN{Kevin Guan}
\IEEEauthorblockN{Cameron Barrie}
\IEEEauthorblockA{\\CS Senior Capstone\\Fall 2016}
}

\maketitle
\begin{abstract}
        People go everywhere with their cell phones in the modern world. If they are not using their phones to make a call, then they are most likely using
        one of its features. Unfortunately, many people are often found disconcerted
        with their cell phones, since unexpected dead zones can be prominent in many locations. 
        Coverage availability plays an integral part of people’s daily lives and routines,
        so they will always need their phones to have service. Our project, the
        Phone Service Alert Application, will focus on relieving the frustrations 
        caused by coverage drops. If somebody's phone loses coverage, the application will
        notify the user when service is available again. The application will also
        integrate detailed coverage maps from service providers,
        allowing the users to view coverage data that is based on their current location and
        guide them to an area with better service. Users will also have
        options to customize their notifications and alerts. Our product will
        bring our users peace of mind, knowing that their phones will notify
        them when service will be available to use again.
\end{abstract}
\pagebreak

\section*{Problem Definition}
The majority of America\textquotesingle s population heavily depends on network coverage for making calls and accessing the Internet. 
The cellphone market in the United States is massive with 92\% of American adults own a cellphone\cite{Zickuhr}, making the demand for reliable service tremendous. 
However, companies have limited resources and cannot always satisfy the masses accordingly. Consequently, the case is common for many users, especially travelers, 
to face abrupt coverage drops and foment heartrending resentments toward their coverage providers. These users must meticulously 
wait and check their mobile devices until their coverage returns before they can return to their work or activities. 
These minutes spent checking for coverage add up to large sums of lost time that could be better invested elsewhere. 

\section*{Proposed Solution}
To implement a solution to this problem, we propose to create develop a phone
service having the capabilities to alert the user before he or she
crosses a service area boundary. To be specific, our solution will be a smart
phone application which is designed to alert the user (as they are in transit)
when he/she is about to enter or exit an area of cellular data coverage for
his/her respective service provider. As an example, take a user who is traveling
as the passenger of a vehicle and is currently receiving no cellular coverage.
Ideally, as the vehicle reaches a predetermined distance from a coverage zone,
the phone will notify the user that it is X distance or X minutes away from
expected service. By default, the application will calculate the distance or
time away from the zone using the vehicle's direction and speed. Alternatively,
we would also like our application to integrate with the phone's navigational
applications to determine the vehicle's route, and use this in calculating time
until a coverage zone is reached. Additionally, the application will also be
able to display a graphical coverage map corresponding to the phone's service
provider. This will allow the user to visually see zones of coverage within his
or her area.\par
One of our more ambitious goals for the application is to allow users to specify
the level of data coverage that they want. That is, they will be able to select
whether to be notified for 2G, 3G, 4G, or LTE data depending on their phone
usage needs (e.g. streaming video, making a call, checking email, etc.). Another
"stretch goal" is to add functionality to the application which will
automatically send pending emails to recipients once coverage zones are reached.
We would also like to disable the phone's periodic search for service when it is
outside a coverage zone. If the phone knows that it is not in a coverage zone,
there is no reason for it to be wasting battery searching for service that isn't
there. However, these goals exist as something to strive for if our basic
solution gets successfully implemented.\par
Our basic solution meets the needs of the problem because it relieves the user
of the hassle of constantly checking his or her phone for coverage. Instead
users can rest assured that their phone will automatically let them know when a
they enter a zone of data coverage. Time previously wasted in repetitive
cellular service checking can now be spent on more important tasks (e.g. keeping
your eyes on the road!). In addition, the coverage maps provided by the
application will provide a coverage searching user with some direction. In other
words, rather than wandering aimlessly in search of cellular service, users will
know exactly where to find the closest coverage zone. In this case, wasted time
searching for cellular service is eliminated.\par
When it comes time to present at the Engineering Expo, our goal is to have a
fully functioning or near fully functioning beta application to show. That is,
we hope to be able to demonstrate the application's coverage map functionality
directly from a smart phone. As the notifications feature of the application will be difficult to demonstrate at Expo without traveling, we would also like
to show off a video or other media of the application in action. Additionally,
we also hope to have a working version of the application on the app store so
that we can encourage others to download it.\par

\section*{Performance Metrics}
The responses from the client's feedback and the quality of fulfillment to the client's requests
will be the key determinants in measuring our performance metrics. The client construed a defined set of objectives 
for us to complete. The project team will meet every week to access current progress and receive assessments and
future expectations from our client. By the end of each meeting, the project team will plan suitable objectives to complete for the following week. 
\par
Since our client has already provided a set of objectives for the team, we will be utlizing them to create an appropriate
plan of actions and milestones to aim for. Currently, we will need our application to be functional with its foundational features that will
allow the app to detect coverage availability from the user's current location. The first 
function the will need to be implemented is an automatic network provider recognition system and 
appropriate map loading. Certainly, if there are no available network providers that can be recognized, 
users can select a provider and load their maps. Secondly, we will also need to implement location 
services. Location services will be crucial for recognizing users’ locations and judging the availability
of coverage wihin the user's viscinity. Thirdly, we will need to integrate the app with notification services. The aim is to 
enable the app to notify the users whenever they enter any area with coverage if their service was ever dropped. 
The last stage will involve receiving feedback on the app's features. We want to establish a small following of loyal users 
to test our app and gather statistics, reviews and testimonials. After we 
received enough information, we can plan accordingly for further improvements and features for 
upcoming versions. When we meet our client's objectives, we will begin working on implementing extra features in newer versions, such as
a feedback system for gathering user criticisms and reviews. 
Our ultimate goal is create an app that has a nice user interface, a notification system that alerts 
users when their dropped service returns, an automatic network provider recognizer, and a reliable coverage map loader. 


\bibliographystyle{IEEEtran}
\bibliography{lesson1}

\noindent\begin{tabular}{ll}\\
\makebox[2.5in]{\hrulefill} & \makebox[2.5in]{\hrulefill}\\
Client & Date\\[8ex]% adds space between the two sets of signatures
\makebox[2.5in]{\hrulefill} & \makebox[2.5in]{\hrulefill}\\
Project Member & Date\\[8ex]
\makebox[2.5in]{\hrulefill} & \makebox[2.5in]{\hrulefill}\\
Project Member & Date\\[8ex]
\makebox[2.5in]{\hrulefill} & \makebox[2.5in]{\hrulefill}\\
Project Member & Date\\[8ex]
\end{tabular}

\end{document}


























