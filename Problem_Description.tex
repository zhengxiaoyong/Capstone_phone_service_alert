\documentclass[letterpaper,10pt,draftclsnofoot,onecolumn]{IEEEtran}

\usepackage{graphicx}                                        
\usepackage{amssymb}                                         
\usepackage{amsmath}                                         
\usepackage{amsthm}                                          

\usepackage{alltt}                                           
\usepackage{float}
\usepackage{color}
\usepackage{url}

\usepackage{balance}
\usepackage[TABBOTCAP, tight]{subfigure}
\usepackage{enumitem}
\usepackage{pstricks, pst-node}

\usepackage{geometry}
\geometry{textheight=9.5 in, textwidth=7.0 in}

\parindent = 0.0 in
\parskip = 0.1 in
\voffset = 0.75 in

\title{Problem Statement}
\date{October 12, 2016}
\author{Cameron Barrie, Kevin Guan, Xiaoyong Zheng}

\begin{document}
\maketitle
\begin{abstract}
        When people are traveling, they are usually using their cell phones or
        one of its features. Unfortunately, they are often left disappointed
        because many areas are dead zones. Phone service availability is vital,
        so they will always check their phones for service. Our project, the
        Phone Service Alert Application, will focus on eliminating this issue by
        notifying users when they enter areas with service. The application will
        be integrated with detailed coverage maps from service providers,
        allowing the users to view the coverage data based on their location.
        Our application will ensure that our users will not have to check their
        phones fanatically for service availability. Users will also have the
        options to customize their notifications and alerts. Our product will
        bring our users peace of mind, knowing that their phones will notify
        them when service will be available again.
\end{abstract}
\pagebreak

% Problem Definition


% Problem Solution
\par
To implement a solution to this problem, we propose to create develop a phone
service having the capabilities to alert the user before he or she
crosses a service area boundary. To be specific, our solution will be a smart
phone application which is designed to alert the user (as they are in transit)
when he/she is about to enter or exit an area of cellular data coverage for
his/her respective service provider. As an example, take a user who is traveling
as the passenger of a vehicle and is currently receiving no cellular coverage.
Ideally, as the vehicle reaches a predetermined distance from a coverage zone,
the phone will notify the user that it is X distance or X minutes away from
expected service. By default, the application will calculate the distance or
time away from the zone using the vehicle's direction and speed. Alternatively,
we would also like our application to integrate with the phone's navigational
applications to determine the vehicle's route, and use this in calculating time
until a coverage zone is reached. Additionally, the application will also be
able to display a graphical coverage map corresponding to the phone's service
provider. This will allow the user to visually see zones of coverage within his
or her area.\par
One of our more ambitious goals for the application is to allow users to specify
the level of data coverage that they want. That is, they will be able to select
whether to be notified for 2G, 3G, 4G, or LTE data depending on their phone
usage needs (e.g. streaming video, making a call, checking email, etc.). Another
"stretch goal" is to add functionality to the application which will
automatically send pending emails to recipients once coverage zones are reached.
We would also like to disable the phone's periodic search for service when it is
outside a coverage zone. If the phone knows that it is not in a coverage zone,
there is no reason for it to be wasting battery searching for service that isn't
there. However, these goals exist as something to strive for if our basic
solution gets successfully implemented.\par
Our basic solution meets the needs of the problem because it relieves the user
of the hassle of constantly checking his or her phone for coverage. Instead
users can rest assured that their phone will automatically let them know when a
they enter a zone of data coverage. Time previously wasted in repetitive
cellular service checking can now be spent on more important tasks (e.g. keeping
your eyes on the road!). In addition, the coverage maps provided by the
application will provide a coverage searching user with some direction. In other
words, rather than wandering aimlessly in search of cellular service, users will
know exactly where to find the closest coverage zone. In this case, wasted time
searching for cellular service is eliminated.\par
When it comes time to present at the Engineering Expo, our goal is to have a
fully functioning or near fully functioning beta application to show. That is,
we hope to be able to demonstrate the application's coverage map functionality
directly from a smart phone. As the notifications feature of the application will be difficult to demonstrate at Expo without traveling, we would also like
to show off a video or other media of the application in action. Additionally,
we also hope to have a working version of the application on the app store so
that we can encourage others to download it.\par

% Performance Metrics


\end{document}


























